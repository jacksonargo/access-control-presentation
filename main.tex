\documentclass{beamer}

\mode<presentation>
{
  \usetheme{default}
  \usecolortheme{default}
  \usefonttheme{default}
  \setbeamertemplate{navigation symbols}{}
  \setbeamertemplate{caption}[numbered]
} 

\usepackage[english]{babel}
\usepackage[utf8]{inputenc}
\usepackage[T1]{fontenc}
\usepackage{lmodern}
\usepackage{amsmath}
\usepackage{amsfonts}
\usepackage{graphicx}
\usepackage{nth}
\usepackage{listings}
\usepackage{color}
\usepackage{hyperref}

\definecolor{dkgreen}{rgb}{0,0.6,0}
\definecolor{gray}{rgb}{0.5,0.5,0.5}
\definecolor{mauve}{rgb}{0.58,0,0.82}

\hypersetup{
    colorlinks=true,
    linkcolor=red,
    urlcolor=cyan
}

\lstset{
  language=bash,
  aboveskip=3mm,
  belowskip=3mm,
  showstringspaces=false,
  columns=flexible,
  basicstyle={\footnotesize\ttfamily},
  numbers=none,
  numberstyle=\tiny\color{gray},
  keywordstyle=\color{blue},
  commentstyle=\color{dkgreen},
  stringstyle=\color{mauve},
  breaklines=true,
  breakatwhitespace=true,
  tabsize=3
}

\title[Access Control]{Access Control}
\author{Jackson Argo}
\institute{Rackspace MO After-hours}

\begin{document}

\begin{frame}
  \titlepage
\end{frame}


\begin{frame}
  \tableofcontents
\end{frame}

\section{What is Access Control?}

\begin{frame}{What is Access Control?}
\begin{itemize}
\item Access control is answering the question: \textit{Who can access what?}
\item Our every day life relies on access control in many ways:
\begin{itemize}
\item We choose which thoughts that we want to share.
\item We carry an id so that we can buy alcohol.
\item The \nth{2} amendment guarantees us access to firearms.
\end{itemize}
\end{itemize}
\end{frame}

\begin{frame}{Access Control in Linux}
Unsurprisingly, Linux relies on access control as well.
\begin{itemize}	
\item The kernel protects memory space so that programs can't inhibit one another.
\item Regular users cannot arbitrarily overwrite the root filesystem. 
\item Apache blocks access to xmlrpc.php so people can run Wordpress sites.
\end{itemize}
\end{frame}

\section{How Is Access Determined?}

\begin{frame}{How Is Access Determined?}
Before we can implement access control, we need to have a reliable way to determine whether access can be permitted. Valid access control requires the following:
\begin{itemize}
\item There must be a non-biased mechanism used to test for access.
\item The mechanism must be sure that the user cannot be spoofed or impersonated.
\item The user must be able to activate this mechanism.
\item The user must be sure that the mechanism cannot be spoofed or impersonated.
\item The result must be enforced by the operating system.
\end{itemize}
\end{frame}

\begin{frame}{Trust}
Like all of security, access control relies on trust. If you use a mechanism for access control, then you inherently trust that the mechanism will work properly.
\end{frame}

\section{Who Determines Access?}

\begin{frame}{Who Determines Access?}
Someone has to determine access control and security of different objects, whether it is an automated program or a user, i.e. \textit{Who has access to control access control?}
\break \break
There are two main approaches:
\begin{itemize}
  \item \textbf{Discretionary} - Any user may be involved in the definition on the policy function and/or assignment of security attributes. This places the burden of security on the user. \textit{E.x. file permissions, Facebook posts, 5$^{th}$ amendment.}
  \item \textbf{Mandatory} - Policy functions and security attributes are tightly controlled by the system administrator. This places the burden of security on the administrator. \textit{E.x. SELinux, firewalls, sudoers.}
\end{itemize}
\end{frame}

\begin{frame}{Discretionary Access Control Gotchas}
\begin{itemize}
\item I want to re-emphasize that \textbf{discretionary access control places the burden of security on the user.} 
\item This means that our customers need to have a good understanding of what needs to be protected and what doesn't.
\item Our customers \textit{are} our customers because they do not necessarily have a good understanding of this.
\item It follows that \textbf{WE} have to be extra vigilant in checking that things like file permissions are sane and will not compromise the server.
\item We should also educate our customers when they are doing something wrong.
\end{itemize}
\end{frame}

\section{Forms of Access Control}

\begin{frame}{Forms of Access Control}
In Linux, the mechanism for access control typically checks the user against predetermined set of criteria.
\break \break
These criteria look like:
\begin {itemize}
\item Simple Yes/No Rules
\item Access Control Lists (ACL's)
\item Role Based Access Control (RBAC)
\end{itemize}
\end{frame}

\begin{frame}{Forms of Access Control}
Here are some common ways we see access control in daily life that are not particularly common in Linux:
\begin{itemize}
\item Attribute Based Access Control (ABAC) - This is a more general form of RBAC. Instead of checking for only a list of roles, you can check any key attribute for any value.
\item Time Based Access Control (TBAC) - Access is only granted during specific times, e.x. department store hours of operation.
\item History Based Access Control (HBAC) - Access is granted based on a history of activities. fail2ban uses this type of access control.
\end{itemize}
\end{frame}

\begin{frame}{Simple Yes/No Rules}
\begin{itemize}
\item Yes/No/Pass/Fail rules are the simplest way to determine access, and all forms of access control can be generalized to these rules.
\item Examples:
\begin{itemize}
\item \texttt{[ \$(id -u) = 0 ] || exit 1}
\item SELinux booleans
\end{itemize}
\end{itemize}
\end{frame}

\subsection{SELinux Booleans}

\begin{frame}{Example: SELinux Booleans}
\lstinputlisting{selinux_booleans.sh}
\end{frame}

\begin{frame}{Access Control Lists}
\begin{itemize}
\item ACL's can be thought of as a chain or flow chart of standardized yes/no rules. The agent will be checked against all the rules until the chain is terminated by a final yes or no.
\item Examples:
\begin{itemize}
\item iptables rules
\item File ACLS
\item Apache access control
\end{itemize}
\end{itemize}
\end{frame}

\subsection{iptables}

\begin{frame}{iptables}
First, we define our iptables chains:
\lstinputlisting[lastline=7]{iptables-rules.sh}
\end{frame}

\begin{frame}{iptables}
Next, we define the rules for each chain:
\lstinputlisting[firstline=8,lastline=19]{iptables-rules.sh}
\end{frame}

\begin{frame}{iptables}
Now we can check our firewall settings. I ran these commands from bastion:
\lstinputlisting{iptables-test.sh}
\end{frame}

\subsection{File Access Control Lists}

\begin{frame}{File Access Control Lists}
\lstinputlisting[lastline=15]{facls.sh}
\end{frame}

\begin{frame}{File Access Control Lists}
\lstinputlisting[firstline=16,lastline=31]{facls.sh}
\end{frame}

\begin{frame}{File Access Control Lists}
\lstinputlisting[firstline=34]{facls.sh}
\end{frame}

\subsection{Apache Access Control}

\begin{frame}{Apache Access Control}
\lstinputlisting[language=XML]{apache_acl.conf}
\end{frame}

\begin{frame}{Apache Access Control}
\lstinputlisting[lastline=7]{apache-test.sh}
\end{frame}

\begin{frame}{Apache Access Control}
\lstinputlisting[firstline=8]{apache-test.sh}
\end{frame}

\begin{frame}{Apache Access Control}
\lstinputlisting[language=XML,lastline=19]{apache-example2.conf}
\end{frame}

\begin{frame}{Role Based Access Control}
Users are assigned roles, and objects have particular roles that are allowed to access them.
These roles are often hierarchical, e.x. full admin > network admin > firewall admin.

Examples:
\begin{itemize}
\item Linux groups
\item Rackspace Cloud RBAC
\item Rackspace internal roles
\end{itemize}
\end{frame}

\subsection{Linux Groups}

\begin{frame}{Linux Groups}
When used in conjunction with sudo, Linux groups are a nice user abstraction that can be extended beyond simple file level access control.

We can use /etc/sudoers to create a \textit{firewall-admin} group that can:
\begin{itemize}
\item Add and remove entries to /etc/hosts.
\item Check what process are listening on open ports.
\item Manage network interfaces and routes.
\item View/update the firewall.
\end{itemize}
\end{frame}

\begin{frame}{Linux Groups}
Here is our sudoers file:
\lstinputlisting{sudoers}
\end{frame}

\begin{frame}{Linux Groups}
Now we put the \textit{firewall-admin} group to use:
\lstinputlisting{sudoers-test}
\end{frame}

\section{Access Control in the Kernel}

\begin{frame}{Access Control in the Kernel}
In each form of access control we've discussed so far, the kernel is the decider and enforcer.
The kernel ultimately decides who has has access to what, and it follows that we should be able to completely isolate a process's access to objects through the kernel alone.
There are two kernel features that give us this power:
\begin{itemize}
\item \textbf{Namespaces} - Isolates a process's virtual resources.
\item \textbf{Control Groups} - Restricts access to hardware resources.
\end{itemize}
\end{frame}

\subsection{Namespaces}

\begin{frame}{Namespaces}
\begin{itemize}
\item Think of namespaces as an abstraction of \textit{chroots}. Instead of just files, we can restrict network interfaces, pid's, users, and more.
\item A  namespace  wraps  a  global system resource in an abstraction that makes it appear to the processes within the namespace that they have their own isolated instance of the global resource.
\item Changes to the global resource are visible to other processes that are members of the namespace, but are invisible to other processes.
\item One use  of  namespaces is to implement containers.
\end{itemize}
\end{frame}

\begin{frame}{Types of Namespaces}
There are 6 types of namespaces:
\begin{itemize}
\item \textbf{Mount} - Essentially a much stronger form of the classic chroot.
\item \textbf{Network} - The namespace gets it's own virtual interface, and cannot see any other interface. Processes can attach to any ports on that interface without interfering with the portmapping outside the namespace.
\item \textbf{PID} - The namespace gets it's own process list. The first process in the namespace will have pid 1.
\end{itemize}
\end{frame}

\begin{frame}{Types of Namespaces}
\begin{itemize}
\item \textbf{User} - Similarly, the namespace gets it's own user list. A user inside the namespace can have the same uid as another user outside the namespace.
\item \textbf{UTS} - Let's you set hostname and domain name for a namespace.
\item \textbf{IPC (Interprocess Communication Mechanisms)} - Isolates message queues, semaphore sets, and shared memory segments.
\end{itemize}
\end{frame}

\subsection{Control Groups}

\begin{frame}{Control Groups}
\begin{itemize}
\item Think of control groups as an abstraction on \textit{nice}.
Instead of only limiting CPU priority, we put limits on pretty much every way a process will interact with the hardware.
\item There is a \textbf{LOT} of control we as system administrators have over cgroups, and I could dedicate an entire brown bag on how these work and how we can use them.
\item Fedora has some pretty thorough documentation on this in their \href{https://docs.fedoraproject.org/en-US/Fedora/17/html/Resource_Management_Guide/index.html}{Resource Management Guide}.
\end{itemize}
\end{frame}

\begin{frame}{Types of Control Groups}
\begin{itemize}
\item \textbf{blkio} — this subsystem sets limits on input/output access to and from block devices such as physical drives (disk, solid state, USB, etc.).
\item \textbf{cpu} — this subsystem uses the scheduler to provide cgroup tasks access to the CPU.
\item \textbf{cpuacct} — this subsystem generates automatic reports on CPU resources used by tasks in a cgroup.
\item \textbf{cpuset} — this subsystem assigns individual CPU's (on a multi-core system) and memory nodes to tasks in a cgroup.
\item \textbf{devices} — this subsystem allows or denies access to devices by tasks in a cgroup.
\end{itemize}
\end{frame}

\begin{frame}{Types of Control Groups}
\begin{itemize}
\item \textbf{freezer} — this subsystem suspends or resumes tasks in a cgroup.
\item \textbf{memory} — this subsystem sets limits on memory use by tasks in a cgroup, and generates automatic reports on memory resources used by those tasks.
\item \textbf{net\_cls} — this subsystem tags network packets with a class identifier (classid) that allows the Linux traffic controller (tc) to identify packets originating from a particular cgroup task.
\item \textbf{net\_prio} — this subsystem provides a way to dynamically set the priority of network traffic per network interface.
\item \textbf{ns} — the namespace subsystem. 
\end{itemize}
\end{frame}

\subsection{Containers}

\begin{frame}{Containers}
\begin{itemize}
\item We have all of these cool access control tools like namespaces, control groups, and SELinux policies, but how to we leverage them together?
\item Linux Containers burrito all of these features to give us very strong process isolation (\url{https://linuxcontainers.org/}).
\item This is the backbone of our favorite tool, \href{http://www.docker.com/}{Docker}.
\end{itemize}
\end{frame}

\begin{frame}{END}
\textbf{Questions? Comments? Concerns? Food?}
\vskip 20pt
\href{https://github.com/jacksonargo/access-control-presentation}{Contribute!}
\end{frame}

\end{document}
